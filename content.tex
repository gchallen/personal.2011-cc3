% <wc:start description="Content" max=800>

\stage{A Harvard Professor and Harvard College student walk into a
therapist's office.}

\dialog{Therapist}{What brings you in today?}

\dialog{Professor}{Nobody comes to my office hours!}

\dialog{Therapist}{Why do you think that is?}

\dialog{Professor}{I think some students just feel bashful.}

\dialog{Therapist}{Student, is that true?}

\dialog{Student}{Not really. Sure, I know a few bashful students, but in
general this isn't a terribly bashful place.}

\dialog{Therapist}{Have you thought about attending office hours?}

\dialog{Student}{Sure, sometimes.}

\dialog{Therapist}{Have you ever gone?}

\dialog{Student}{No.}

\dialog{Therapist}{Why not?}

\dialog{Student}{I guess I just haven't had relevant enough questions. I
don't know.}

\dialog{Therapist}{Have you reached out to faculty in other ways?}

\dialog{Student}{Sure, I invited my favorite Professor to the student-faculty
dinner at my house.}

\dialog{Therapist}{How did that go?}

\dialog{Student}{Just OK. It was awkward, really. The Professor I invited
didn't really seem to know much about my life and wasn't that curious.}

\dialog{Professor}{Can I respond?}

\dialog{Therapist}{Sure.}

\dialog{Professor}{Students seem distant to us too. They only care about
their grade and few  want to engage deeply with the material. And they're too
busy to! Clubs, societies, publications, sports, choirs, orchestras, you name
it. They're all doing fifty other things and make it very clear to you that
your class is their last priority.}

\dialog{Student}{OK, since we're being really honest, Professors around here
make their priorities very clear! Talk about busy! First it's research,
grant-writing, and supervising their graduate students who they really care
about. After that, they want to start companies and consult to augment their
already generous salaries, appear on television, write books, and so on. What
do I get for my insane tuition? Last year's warmed-over lectures and two
office hours a week.}

\dialog{Therapist}{It's clear you're both upset. Let's step back for a
minute. Professor, do you feel like there are opportunities to meet students
on campus?}

\dialog{Professor}{Definitely! Harvard is great at that. I can take meals in
the dining hall, attend sporting or other extracurricular events, and even
join one of the House Senior Common Rooms or become a House Master. I even
heard a Professor who applied to be a Resident Tutor.}

\dialog{Student}{Really? What happened to him?}

\dialog{Professor}{Nobody hired him! I think they thought he was a bit
crazy.}

\dialog{Therapist}{Professor, have you tried any of these things?}

\dialog{Professor}{No, not really. I went to lunch once but nobody talked to
me.}

\dialog{Therapist}{Would it be possible to try again? Not all students are
great conversationalists.}

\dialog{Professor}{Nobody cares! I'm up for tenure in three years. I have to
crank out publications and degrees. I don't get any credit for any of this
student contact stuff, everyone knows that! My only interaction with the
College that matters is having halfway-decent CUE scores, and as long as I
crack a few jokes here and there and grade generously I'm set. I want to
spend more time with undergraduates. It's just not part of my job
description.}

\dialog{Student}{I mean, I want to believe you, but I've had some really
inspiring teachers here. David Malan. Tim McCarthy. They seem to be able to
do it.}

\stage{Professor looks uncomfortable.}

\dialog{Professor}{Yeah, those guys are great, but they don't do any research
and don't have full appointments.}

\dialog{Therapist}{Perhaps research is the problem then?}

\dialog{Professor}{No, no! Research is great! It's the reason I'm here, and
it's what makes Harvard the best.}

\dialog{Student}{But I came here for an education. So while you're busy in
the lab, I'm learning from my peers, in those clubs, teams, and
extracurricular organizations you mentioned earlier.}

\dialog{Therapist}{Professor, do you think things will change after you get
tenure?}

\dialog{Professor}{Yes! Then I'll be able to completely change the values
that have been instilled in me during my first six years here. I'll
definitely spend more time with students.}

\dialog{Student}{C'mon. Isn't after tenure the time when you slow down, spend
more time with your family, and start a company on the side?}

\dialog{Professor}{That's not fair.}

\stage{The two sit in silence for a minute.}

\dialog{Therapist}{Let me summarize. Professor, you claim you'd like to spend
more time with students but the research orientation of Harvard prevents you
from doing so. Certainly there are no good incentives for this behavior.
Student, you feel ignored by faculty generally but have responded by building
your own community centered around interacting with your peers through
extracurricular activities. You're happy for the main part but wonder how
much of your tuition is going to fund the community of scholars who seem
largely uninterested in you and a growing number of administrators hired to
pay attention to you when Professors don't.}

\dialog{Professor}{That's fair, I guess. I didn't design the system, but it's
the one we have.}

\dialog{Student}{Sure, that's about right.}

\dialog{Therapist}{I can't say that this seems like a healthy relationship.
I think you two have reached a stable but unhappy cohabitation. Let me make a
suggestion.}

\dialog{Professor}{Sure.}

\dialog{Therapist}{You two need to decide whether you really want this
relationship to work.}

% <wc:end>

\textit{Geoffrey Challen '02--'03 is a Resident Tutor at Eliot House. The
views expressed are his and do not reflect official Harvard College policy.}
